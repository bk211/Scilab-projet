\documentclass[12pt, letterpaper]{article}
\usepackage[utf8]{inputenc}
\usepackage{graphicx} % inserer des figures
\usepackage{hyperref} % lien html
\usepackage{fancyvrb} % Vertbatim avec une majuscule
\usepackage{array} % tableaux

\title{Traitement du signal et des images : Projet mini boîte a outils pour traitement d'image}
\author{Chaolei Cai
\\
    \multicolumn{1}{
        p{.7\textwidth}}{\centering\emph{Université Paris Vincennes St-Denis\\
  UFR mathématiques, informatique, technologies sciences de l'information\\}
  L3 Informatique}
}
\date{\today}
\begin{document}


\begin{titlepage}
    \maketitle
\end{titlepage}

\tableofcontents

\section{Présentation}
Ce document est mon rapport pour le cours de Traitement du signal et des images enseigné par M.Boubchir au 3ème année de 
la Licence Informatique.\\
Le but du projet est donc crée un petit programme avec une interface \\graphique, permettant ainsi de 
visualiser rapidement certain traitement applicable sur l'image.

\section{Dépendances}
Le projet a été écrite en Scilab 6.0.2, il nécessite la librairie IPCV (Image Processing and Computer Vision Toolbox).\\
L'interface graphique a été écrite via l'outils guibuilder, si vous rencontrez des problèmes d'exécution, vérifier la version de Scilab 
que vous possèdez et vérifier la présence de ces 2 librairies. Recemment, il y a eu une mise à jour de Scilab en 6.1.0, je n'ai 
pas encore tester le programme sous cette version car le developpement a été faite en 6.0.2\\
La compatibilité avec une version Scilab 5.X.X est sans doute possible, mais il faudra charger le module 
SIVP (Scilab Image and Processing Video).






%\begin{Verbatim}[numbers=left,xleftmargin = 5mm]


\end{document}

